\documentclass[a4paper,10pt]{report}
\usepackage[utf8]{inputenc}

% Title Page
\title{Analyse de Comportements avec Twitter}
\author{Antonin Durey Matthieu Caron}


\begin{document}
\maketitle

\chapter{Description générale du projet}
  \section{Description de la problématique}
    L'idée c'est de créer une application permettant de tester différents algorithmes
    afin de faire une analyse de sentiments sur twitter. La problématique est donc la suivante,
    quel est l'algorithme le plus efficace (qui donne le plus souvent la vérité) pour faire de la 
    classification de sentiments?
  \section{Description générale de l'architecture de votre application}
    Nous avons fait une application java swing. Voici notre diagramme de classes.
    %%
    %%	todo diagramme de classes
    %%
\chapter{Détails des différents travaux réalisés}
  Le projet se divise en quatre taches majeurs, utilisation de une API twitter afin de récupérer 
  des tweets, préparer une base d'apprentissage afin de ne pas avoir de bruit mais aussi de créer notre 
  base de vérité pour l'apprentissage supervisé. Implémenter et tester différents algorithmes de classification de sentiments et
  enfin pouvoir utiliser ses algorithme depuis une application avec son interface graphique.
  \section{API Twitter}
  \section{Préparation de la base d'apprentissage}
    \subsection{Netoyage des données}
    \subsection{Construction de la base}
  \section{Algorithme de classification}
    \subsection{mots clefs}
    \subsection{KNN}
    \subsection{Bayes}
  \section{Interface graphique}
    \subsection{copie d'ecran}
    \subsection{manuel d'utilisation}
\chapter{Résultats de la classification avec les différentes méthodes et analyse}
\chapter{Conclusions}




\end{document}          
